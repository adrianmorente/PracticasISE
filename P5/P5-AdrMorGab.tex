\input{preambuloSimple.tex}

%----------------------------------------------------------------------------------------
%	TÍTULO Y DATOS DEL ALUMNO
%----------------------------------------------------------------------------------------

\title{
\normalfont \normalsize
\textsc{\textbf{Ingeniería de Servidores (2016-2017)} \\ Grado en Ingeniería
Informática \\ Universidad de Granada} \\ [25pt] % Your university, school and/or department name(s)
\horrule{0.5pt} \\[0.4cm] % Thin top horizontal rule
\huge Memoria Práctica 5 \\ % The assignment title
\horrule{2pt} \\[0.5cm] % Thick bottom horizontal rule
}

\author{Adrián Morente Gabaldón} % Nombre y apellidos

\date{\normalsize\today} % Incluye la fecha actual

%----------------------------------------------------------------------------------------
% DOCUMENTO
%----------------------------------------------------------------------------------------

\begin{document}

\maketitle % Muestra el Título

\newpage %inserta un salto de página

\tableofcontents % para generar el índice de contenidos

\newpage

\listoffigures

\listoftables

\newpage


\section{[SYSCTL] Al modificar los valores del kernel de este modo, no logramos que persistan después de reiniciar
la máquina. ¿Qué archivo hay que editar para que los cambios sean permanentes?}
Como ya se explicó en clase de prácticas, \emph{Sysctl} modifica los parámetros del kernel en tiempo de ejecución,
por lo que al reiniciar la máquina se pierden los valores modificados. Si ejecutamos Sysctl a través de línea de
comandos sin opciones, se nos despliega una pequeña lista con sus opciones más destacadas, que son las siguientes:
\begin{figure}[H]
	\centering
	\includegraphics[scale=0.6]{sysctl-h}
	\caption{Opciones más destacadas de sysctl. - Adrián Morente Gabaldón [26/12/2016]}
	\label{figura1}
\end{figure}
Como vemos en la captura de pantalla anterior, el propio sysctl nos redirige a su manual si queremos explorar más
opciones u obtener más información. Al principio de este manual, encontramos que todos los parámetros configurables
descienden del directorio \emph{/proc/sys}, ordenados en subcarpetas según pertenencia (sistema de archivos, kernel,
memoria virtual, etc), y cada uno de ellos se encuentra en formato de archivo en texto plano, conteniendo tan solo
el valor del parámetro en cuestión. Veamos un ejemplo de los parámetros pertenecientes al módulo de memoria virtual:
\begin{figure}[H]
	\centering
	\includegraphics[scale=0.7]{proc-sys}
	\caption{Contenido del directorio /proc/sys, sus subdirectorios y ejemplo de uno de sus parámetros. - Adrián
	Morente Gabaldón [28/12/2016]}
	\label{figura3}
\end{figure}
Para que los cambios persistan tras reiniciar la máquina, debemos aplicar la modificación a cada uno de los ficheros
de parámetros. Sin embargo, por temas de seguridad, es mejor utilizar esta herramienta con algunas de sus múltiples
opciones en lugar de acceder y modificar directamente dichos ficheros (ya que podemos ``tocar donde no debemos''),
cosa que podría derivar en algún fallo no deseado del sistema.
Como vimos en clase, y como bien comenta el manual de \emph{sysctl}, la configuración perteneciente y valorable por
esta herramienta se encuentra principalmente en el archivo /etc/sysctl.config, y colgando del directorio /etc/sysctl.d.
Veamos una parte del contenido de dicho primer archivo:
\begin{figure}[H]
	\centering
	\includegraphics[scale=0.36]{sysctl-conf}
	\caption{Parte del contenido del archivo de configuración /etc/sysctl.config. - Adrián Morente Gabaldón [19/01/2017]}
	\label{figura5}
\end{figure}
Como podemos apreciar, no encontramos mucha información sobre qué es cada cosa, solo encontramos nombres de variables
con sus correspondientes valores; todos ellos ordenados de forma clara y precisa según su ámbito. Por ejemplo, veamos
uno de estos ámbitos que, personalmente, me ha llamado la atención, y es el relacionado con la \textbf{seguridad} del sistema:
\begin{figure}[H]
	\centering
	\includegraphics[scale=0.45]{sysctl-security}
	\caption{Contenido del archivo de configuración /etc/sysctl.config relacionado con la seguridad del sistema. - Adrián
	Morente Gabaldón [19/01/2017]}
	\label{figura6}
\end{figure}
En este apartado, encontramos parámetros configurables que nos permitirían en cierto modo evitar algunos ataques a nuestro
sistema, como pueden ser el bloqueo de redirecciones mediante el protocolo ICMP (que incluye herramientas como ping
o traceroute, como hemos visto en la asignatura de \emph{Fundamentos de Redes}). Como bien explica la pequeña introducción
en este apartado, estas son medidas contra el \emph{spoofing} (que en español se traduce por \emph{burla} o \emph{engaño},
y en informática entendemos por ``falsificación de identidad'') y contra ataques \emph{Man In The Middle}, término que ya
conocemos.\\
Para terminar, cabe destacar que todos estos últimos parámetros están comentados, de forma que el sistema toma valores por
defecto en caso de que no sean modificados aquí. Las instrucciones del archivo nos instan a no modificar parámetros si no
sabemos lo que estamos haciendo. Además, ya sabemos que en caso de tener que modificarlos, debemos hacer copia de seguridad
previa a su modificación.

\section{¿Con qué opción se muestran todos los parámetros modificables en tiempo de ejecución? Elija dos parámetros
y explique, en dos líneas, qué función tienen.}
Si leemos el manual de sysctl en la terminal, vemos rápidamente que la opción para consultar todas las variables
modificables en ejecución es \emph{-a}:
\begin{figure}[H]
	\centering
	\includegraphics[scale=0.7]{sysctl-a}
	\caption{Opción de Sysctl para consultar los parámetros modificables en ejecución. - Adrián Morente Gabaldón [26/12/2016]}
	\label{figura2}
\end{figure}
Si ejecutamos \textbf{sysctl -a} obtenemos una extensa lista con todas las variables configurables. Exactamente,
tantas como ficheros había en los subdirectorios de \emph{/proc/sys} vistos en el ejercicio anterior, lógicamente.

\section{[Windows Server] a) Realice una copia de seguridad del registro y restáurela, ilustre el proceso con capturas.
b) Abra una ventana mostrando el editor del registro.}
Para empezar, seguiremos las instrucciones dictadas por el guión de prácticas, comenzando por ejecutar \emph{regedit}
desde la línea de comandos de Windows Server. A continuación, nos encontraremos con esta ventana:
\begin{figure}[H]
	\centering
	\includegraphics[scale=0.7]{regedit}
	\caption{Ventana principal del Editor del Registro en Windows Server. - Adrián Morente Gabaldón [26/12/2016]}
	\label{figura4}
\end{figure}


\section{Enumere qué elementos se pueden configurar en Apache y en IIS para que Moodle funcione mejor.}


\section{Ajuste la compresión en el servidor y analice su comportamiento usando varios valores para el tamaño de archivo
a partir del cual comprimir. Para comprobar que está comprimiendo puede usar el navegador o comandos como curl (see url)
o lynx. Muestre capturas de pantalla de todo el proceso.}


\section{a) Usted parte de un SO con ciertos parámetros definidos en la instalación (Práctica 1), ya sabe instalar
servicios (Práctica 2) y cómo monitorizarlos (Práctica 3) cuando los somete a cargas (Práctica 4). Al igual que ha visto
cómo se puede mejorar un servidor web (Práctica 5 Sección 3.1), elija un servicio (el que usted quiera) y modifique un
parámetro para mejorar su comportamiento. b) Monitorice el servicio antes y después de la modificación del parámetro
aplicando cargas al sistema (antes y después) mostrando los resultados de la monitorización.}

\section{PREGUNTAS OPCIONALES}
	\subsection{Realice lo mismo que en la cuestión 6 pero para otro servicio.}

\section{PREGUNTAS OPCIONALES DE PRÁCTICAS ANTERIORES}



\bibliography{citas}
\bibliographystyle{plain}
\end{document}
